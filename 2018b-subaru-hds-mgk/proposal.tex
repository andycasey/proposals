\documentclass{article}
\usepackage{subaru}

\begin{document}

\semester{}
\proposalid{}
\receivedate{}

%%%%%%%%%%%%
%% page 1 %%
%%%%%%%%%%%%

%%%%%%%%%% 1. Title of Proposal %%%%%%%%%%
\title{The origin of peculiar Mg/K stars throughout the Milky Way [AUS time] }

%%%%%%%%%% 2. Principal Investigator's information %%%%%%%%%%
\PIfirstname {Andrew}
\PIlastname  {Casey}
\PIinstitute {Monash University}
\PIaddress   {School of Physics and Astronomy, Monash University, Wellington Rd, Clayton VIC 3800 Australia}
\PIemail     {andrew.casey@monash.edu}
\PIphone     {+61 431 296 185}

%%%%%%%%%% 3. Scientific Category %%%%%%%%%%
% Uncomment ONE of the following lines to indicate 
% the scientific category
%\SolarSystem
\NormalStars
%\ExtrasolarPlanets
%\StarandPlanetFormation
%\MetalPoorStars
%\CompactObjectsandSNe
%\MilkyWay
%\LocalGroup
%\ISM
%\NearbyGalaxies
%\AGNandQSOActivity
%\QSOAbsorptionLinesandIGM
%\Cosmology
%\GravitationalLenses
%\ProtoClustersandGalaxyEnvironment
%\ClustersofGalaxies
%\HighzGalaxiesLAEsLBGs
%\HighzGalaxiesothers
%\Miscellaneous

%%%%%%%%%% 4. Abstract %%%%%%%%%%
\begin{abstract}
\end{abstract}

%%%%%%%%%% 5. Co-Investigators %%%%%%%%%%
%\CoI{first name}{last name}{institute}{email address}
\begin{investigators}
\CoI{Kemp}{Alexander}{Monash University}{ajkem1@student.monash.edu}
%\CoI{Ho}{Anna}{Caltech}{ah@caltech.edu}
%\CoI{Ness}{Melissa}{Columbia University}{melissa.ness@columbia.edu}
%\CoI{Ji}{Alexander}{Carnegie Observatories}{aji@carnegiescience.edu}
\CoI{Miles}{Matthew}{Monash University}{mtmil3@student.monash.edu}
\CoI{Norfolk}{Brodie}{Monash University}{bjlee7@student.monash.edu}
\CoI{}{}{}{}
\CoI{}{}{}{}
\CoI{}{}{}{}
\CoI{}{}{}{}
\CoI{}{}{}{}
%no need to fill \coiflag
\coiflag{}

%%%%%%%%%% 6. Thesis Work %%%%%%%%%%
%\thesis{student name}{thesis title}
\thesis{}{}

%%%%%%%%%% 7. Subaru Open Use Intensive Program %%%%%%%%%%
% Uncomment this line if this is an Open Use Intensive Program
%\intensive

\end{investigators}

%%%%%%%%%%%%
%% page 2 %%
%%%%%%%%%%%%


%%%%%%%%%% 8. List of Applicants' Related Publications %%%%%
\newcounter{PublicationNumber}\addtocounter{PublicationNumber}{1}
\newcommand{\publication}[5]{[\arabic{PublicationNumber}] #1 \emph{#2} \textbf{#3} #4 (#5).\stepcounter{PublicationNumber}}
\newcommand\apj{Astrophys. J.}
\newcommand\apjl{Astrophys. J. Lett.}
\newcommand\apjs{Astrophys. J. Supp.}
\newcommand\nature{Nature}
\newcommand\mnras{Mon. Not. Roy. Ast. Soc.}
\newcommand\pasa{Pub. of the Ast. Soc. of Aust.}
\begin{publications}
% Here I have listed anything where we are within first few authors,
% and the results are derived from spectroscopy.
\publication{Casey \& Schlaufman}{\apj}{850}{179}{2017}
\publication{Casey et al.}{\apj}{840}{59}{2017}
\publication{Hogg, Casey, Ness et al.}{\apj}{833}{262}{2016}
\publication{Casey et al.}{\mnras}{461}{3336}{2016}
\publication{Howes, Casey et al.}{\nature}{527}{484}{2015}
\publication{Casey}{\apjs}{223}{8}{2016}
\publication{Koposov, Casey et al.}{\apj}{811}{62}{2015}
\publication{Casey \& Schlaufman}{\apj}{809}{110}{2015}
\publication{Casey et al.}{\mnras}{443}{828}{2014}
\publication{Casey et al.}{\apj}{784}{19}{2014}
\publication{Keller, Bessell, Frebel, Casey et al.}{\nature}{506}{463}{2014}
\publication{Casey et al.}{\apj}{764}{39}{2013}
% Ness:
\publication{Ness, Hogg, Casey et al.}{\apj}{853}{198}{2018}
\publication{Garcia Perez, Ness et al.}{\apj}{852}{91}{2018}
\publication{Ness}{\pasa}{35}{3}{2018}
\publication{Zasowski, Ness et al.}{\apj}{832}{132}{2016}
\publication{Ness \& Freeman}{\pasa}{33}{22}{2016}
\publication{Ness et al.}{\apj}{823}{114}{2016}
\publication{Ness et al.}{\apj}{819}{2}{2016}
\publication{Ness et al.}{\apj}{808}{16}{2015}
\publication{Ness et al.}{\apjl}{787}{19}{2014}
\publication{Ness, Asplund \& Casey}{\mnras}{445}{2994}{2014}
\publication{Ness et al.}{\mnras}{432}{2092}{2013}
\publication{Ness et al.}{\mnras}{430}{836}{2013}
\publication{Ness et al.}{\apj}{756}{22}{2013}
% Ho:
\publication{Ho, Rix, Ness et al.}{\apj}{841}{40}{2017}
\publication{Ho, Ness et al.}{\apj}{836}{5}{2017}
% Ji
\publication{Ji \& Frebel}{\apj}{}{accepted (arXiv:1802.07272)}{2018}
\publication{Safarzadeh, Ji et al.}{\mnras}{}{accepted (arXiv:1712.03967)}{2017}
\publication{Ji, Frebel, Ezzeddine, \& Casey}{\apjl}{832}{3}{2016}
\publication{Ji et al.}{\apj}{830}{93}{2016}
\publication{Ji et al.}{\nature}{531}{610}{2016}
\publication{Ji et al.}{\apj}{817}{41}{2016}
\publication{Ji et al.}{\mnras}{454}{659}{2015}
\publication{Frebel, Chiti, Ji et al.}{\apjl}{810}{27}{2015}
\publication{Ji et al.}{\apj}{782}{95}{2014}
\end{publications}

%%%%%%%%%% 9. Condition of Closely-Related Past and Scheduled Observations %%%%%
%\relatobs{proposal ID}{title}{observational condition}{achievement(%)}
\begin{relationto}
\relatobs{None}{}{}{}
\relatobs{}{}{}{}
\relatobs{}{}{}{}
\relatobs{}{}{}{}
\relatobs{}{}{}{}
\end{relationto}

%%%%%%%%%% 10. Post-Observation Status and Publications %%%%%%%%%%
%\pastrun{year/month}{proposal ID}{PI name}{status:data reduction/analysis}{status:publication}
\begin{previoususe}
\pastrun{None}{}{}{}{}
\pastrun{}{}{}{}{}
\pastrun{}{}{}{}{}
\pastrun{}{}{}{}{}
\pastrun{}{}{}{}{}
\pastrun{}{}{}{}{}
\end{previoususe}

%%%%%%%%%% 11. Experience %%%%%%%%%%
\experience{PI Casey and CIs Ji and Ness are experts in high-resolution stellar spectroscopy, and CI Ho is an expert in low-resolution stellar spectroscopy. The remaining CIs are students. PI Casey, who will prepare and conduct the observations, has observed with the following telescope/instrument combinations:  Magellan/MIKE, VLT/UVES, Keck/HIRES, Gemini North/GMOS-N and Gemini South/GMOS-S (service mode), Mayall telescope/echelle, AAT/AAOmega, Automated Planet Finder (2.5 m) telescope, the ANU 2.3 m telescope, and the 2.2 m MPG/ESO telescope, and about a month of commissioning time with the SkyMapper (imaging) telescope. Casey has written bespoke \texttt{Python} reduction packages for several of these telescope/instrument combinations, and investigators have all written software for the analysis of both low- and high-resolution spectra (Ho, Ness, and Casey; Ji, Ness, and Casey, respectively). Casey has served as an expert reviewer for time allocation committees for Australia and Canada.}


%%%%%%%%%%%%
%% page 3 %%
%%%%%%%%%%%%

%%%%%%%%%% 12. Observing Run %%%%%%%%%%
%\run{instrument}{# of nights}{lunar phase}{preferred dates}{acceptable dates}{observing mode}
% "totnights" is calcurated automatically by ProMS. No need to fill \totnights{}.
\begin{observingrun}
\run{}{}{}{}{}{}
\run{}{}{}{}{}{}
\run{}{}{}{}{}{}
\minnights{0}
\totnights{0}
\secondchoice{}
\orcomment{}
\end{observingrun}

%%%%%%%%%% 13. Scheduling Requirements %%%%%%%%%%
\begin{schedule}
\scheduling{}
% Uncomment the following line if you want to make remote observations at Hilo. 
%\remoteobs
% Uncomment the following line if you want to make remote observations at Mitaka. 
%\remotemtk
% Uncomment the following line if you want to make ToO observations. 
%\obstoo
% Uncomment the following line if you want to make time critial observations. 
%\obstcritical
\end{schedule}

%%%%%%%%%% 14. List of Targets %%%%%%%%%%
%\target{name}{RA}{DEC}{magnitude}
% Equinox is J2000.0 unless otherwise specified in Comments on Targets
% HH MM SS +/-DD MM SS or HH:MM:SS +/-DD:MM:SS are recommended for format of RA and Dec
\begin{targets}
\target{}{}{}{}
\target{}{}{}{}
\target{}{}{}{}
\target{}{}{}{}
% no need to fill \targetflag
\targetflag{}
\targetcomment{}
\end{targets}


%%%%%%%%%%%%
%% page 4 %%
%%%%%%%%%%%%

%%%%%%%%%% 15. Observing Method and Technical Details %%%%%%%%%%
\begin{technicalinfo}
\end{technicalinfo}

%%%%%%%%%% 16. Instrument Requirements %%%%%%%%%%
\instruments{}

%%%%%%%%%% 17. Backup Proposal in Poor Conditions %%%%%%%%%%
\backup{}

%%%%%%%%%% 18. Public Data Archive of Subaru %%%%%
\begin{smoka}
% Uncomment the following line if you checked the public Subaru Telescope archive. 
%\smokacheck
% If you propose observations in spite that there are available data in the public data archive, 
% please denote the reason in the following field. 
\smokacomment{}
\end{smoka} 

%%%%%%%%%% 19. Justify Duplications with the HSC SSP %%%%%%%%%%
\hscssp{}


%%%%%%%%%%%%%
%% page 5+ %%
%%%%%%%%%%%%%

%%%%%%%%%% 14+. List of (more) Targets  %%%%%%%%%%
%*** (optional) ***
% If you have more targets, please uncomment the following 
% 3 lines and enter them here. 
%\begin{moretargets}
%\target{}{}{}{}
%\end{moretargets}

%%%%%%%%%% 17+. List of Backup Targets %%%%%%%%%%
%*** (optional) ***
% If you have backup targets decribed in entry 17, 
% please uncomment the following 3 lines and enter their data here, 
% using the same format as in entry 14. 
%\begin{backuptargets}
%\target{}{}{}{}
%\butargetcomment{}
%\end{backuptargets}


\end{document}

