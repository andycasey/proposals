\documentclass{article}

\usepackage{subaru}
\usepackage{soul}
%\usepackage{color}
\begin{document}


\newcommand{\todo}[1]{\ul{\MakeUppercase{\textbf{#1}}}}
\newcommand{\NumStars}{30}
\newcommand{\NumNights}{two grey}
\semester{}
\proposalid{}
\receivedate{}

%%%%%%%%%%%%
%% page 1 %%
%%%%%%%%%%%%

%%%%%%%%%% 1. Title of Proposal %%%%%%%%%%
\title{The origin of peculiar Mg/K stars throughout the Milky Way [AUS time] }

%%%%%%%%%% 2. Principal Investigator's information %%%%%%%%%%
\PIfirstname {Andrew}
\PIlastname  {Casey}
\PIinstitute {Monash University}
\PIaddress   {School of Physics and Astronomy, Monash University, Wellington Rd, Clayton VIC 3800 Australia}
\PIemail     {andrew.casey@monash.edu}
\PIphone     {+61 431 296 185}

%%%%%%%%%% 3. Scientific Category %%%%%%%%%%
% Uncomment ONE of the following lines to indicate 
% the scientific category
%\SolarSystem
\NormalStars
%\ExtrasolarPlanets
%\StarandPlanetFormation
%\MetalPoorStars
%\CompactObjectsandSNe
%\MilkyWay
%\LocalGroup
%\ISM
%\NearbyGalaxies
%\AGNandQSOActivity
%\QSOAbsorptionLinesandIGM
%\Cosmology
%\GravitationalLenses
%\ProtoClustersandGalaxyEnvironment
%\ClustersofGalaxies
%\HighzGalaxiesLAEsLBGs
%\HighzGalaxiesothers
%\Miscellaneous

%%%%%%%%%% 4. Abstract %%%%%%%%%%
\begin{abstract}
Stars with unusual chemical abundances offer clues about rare astrophysical events or nucleosynthetic pathways. One such example are stars with significantly depleted magnesium and enhanced potassium ($[{\rm Mg/Fe}] < -1$; $[{\rm K/Fe}] > 1$). Stars with this strong signature have, to date, only been found in the globular cluster NGC~2419. The origin of this pattern remains unknown. Theoretical attempts to reproduce it require unrealistic temperature conditions and nuclear cross-sections orders of magnitude above laboratory measurements. Stars in NGC~2419 show a correlation between the abundances of light (e.g. Mg, K) and neutron-capture elements, which could point to the responsible mechanism, but so far only 5 Mg/K stars have detailed abundances measured. Chemical abundances are needed for more stars showing the Mg/K pattern to understand its peculiar origin. We have identified 112 Milky Way field stars with significantly enhanced [K/Fe] and depleted [Mg/Fe] from $\sim$500,000 giant stars in LAMOST. Our sample suggests that the mechanism responsible for peculiar abundances in NGC 2419 is not limited to globular clusters. We seek \NumNights\ nights with Subaru/HDS to observe \NumStars\ candidates in order to measure detailed abundances and understand the origin of this unexplained signature.
\end{abstract}

%%%%%%%%%% 5. Co-Investigators %%%%%%%%%%
%\CoI{first name}{last name}{institute}{email address}
\begin{investigators}
\CoI{Kemp}{Alexander}{Monash University}{ajkem1@student.monash.edu}
\CoI{Miles}{Matthew}{Monash University}{mtmil3@student.monash.edu}
\CoI{Norfolk}{Brodie}{Monash University}{bjlee7@student.monash.edu}
\CoI{Lattanzio}{John}{Monash University}{john.lattanzio@monash.edu}
\CoI{Karakas}{Amanda}{Monash University}{amanda.karakas@monash.edu}
\CoI{Schlaufman}{Kevin}{Johns Hopkins University}{kschlaufman@jhu.edu}
\CoI{Ho}{Anna}{Caltech}{ah@caltech.edu}
\CoI{Tout}{Christopher}{University of Cambridge}{cat@ast.cam.ac.uk}
\CoI{Ness}{Melissa}{Columbia University}{melissa.ness@columbia.edu}
\CoI{Ji}{Alexander}{Carnegie Observatories}{aji@carnegiescience.edu}
%\CoI{}{}{}{}
%\CoI{}{}{}{}
%\CoI{}{}{}{}
%\CoI{}{}{}{}
%no need to fill \coiflag
%\coiflag{}
%%%%%%%%%% 6. Thesis Work %%%%%%%%%%
%\thesis{student name}{thesis title}
%\thesis{}{}
%%%%%%%%%% 7. Subaru Open Use Intensive Program %%%%%%%%%%
% Uncomment this line if this is an Open Use Intensive Program
%\intensive
\end{investigators}

%%%%%%%%%%%%
%% page 2 %%
%%%%%%%%%%%%


%%%%%%%%%% 8. List of Applicants' Related Publications %%%%%
\newcounter{PublicationNumber}\addtocounter{PublicationNumber}{1}
\newcommand{\publication}[5]{[\arabic{PublicationNumber}] #1 \emph{#2} \textbf{#3} #4 (#5).\stepcounter{PublicationNumber}}
\newcommand\apj{Astrophys. J.}
\newcommand\apjl{Astrophys. J. Lett.}
\newcommand\apjs{Astrophys. J. Supp.}
\newcommand\nature{Nature}
\newcommand\mnras{Mon. Not. Roy. Ast. Soc.}
\newcommand\pasa{Pub. of the Ast. Soc. of Aust.}
\begin{publications}
% Here I have listed anything where we are within first few authors,
% and the results are derived from spectroscopy.
\publication{Casey \& Schlaufman}{\apj}{850}{179}{2017}
\publication{Casey et al.}{\apj}{840}{59}{2017}
\publication{Hogg, Casey, Ness et al.}{\apj}{833}{262}{2016}
\publication{Casey et al.}{\mnras}{461}{3336}{2016}
\publication{Howes, Casey et al.}{\nature}{527}{484}{2015}
\publication{Casey}{\apjs}{223}{8}{2016}
\publication{Koposov, Casey et al.}{\apj}{811}{62}{2015}
\publication{Casey \& Schlaufman}{\apj}{809}{110}{2015}
\publication{Casey et al.}{\mnras}{443}{828}{2014}
\publication{Casey et al.}{\apj}{784}{19}{2014}
\publication{Keller, Bessell, Frebel, Casey et al.}{\nature}{506}{463}{2014}
\publication{Casey et al.}{\apj}{764}{39}{2013}
% Karakas:
\publication{Iliadis, Karakas et al.}{\apj}{818}{98}{2016}
% Ho:
\publication{Ho, Rix, Ness et al.}{\apj}{841}{40}{2017}
\publication{Ho, Ness et al.}{\apj}{836}{5}{2017}
% Ness:
\publication{Ness, Hogg, Casey et al.}{\apj}{853}{198}{2018}
\publication{Garcia Perez, Ness et al.}{\apj}{852}{91}{2018}
\publication{Ness}{\pasa}{35}{3}{2018}
\publication{Zasowski, Ness et al.}{\apj}{832}{132}{2016}
\publication{Ness \& Freeman}{\pasa}{33}{22}{2016}
\publication{Ness et al.}{\apj}{823}{114}{2016}
\publication{Ness et al.}{\apj}{819}{2}{2016}
\publication{Ness et al.}{\apj}{808}{16}{2015}
\publication{Ness et al.}{\apjl}{787}{19}{2014}
\publication{Ness, Asplund \& Casey}{\mnras}{445}{2994}{2014}
\publication{Ness et al.}{\mnras}{432}{2092}{2013}
\publication{Ness et al.}{\mnras}{430}{836}{2013}
\publication{Ness et al.}{\apj}{756}{22}{2013}
% Ji
\publication{Ji \& Frebel}{\apj}{}{accepted (arXiv:1802.07272)}{2018}
\publication{Safarzadeh, Ji et al.}{\mnras}{}{accepted (arXiv:1712.03967)}{2017}
\publication{Ji, Frebel, Ezzeddine, \& Casey}{\apjl}{832}{3}{2016}
\publication{Ji et al.}{\apj}{830}{93}{2016}
\publication{Ji et al.}{\nature}{531}{610}{2016}
\publication{Ji et al.}{\apj}{817}{41}{2016}
\publication{Ji et al.}{\mnras}{454}{659}{2015}
\publication{Frebel, Chiti, Ji et al.}{\apjl}{810}{27}{2015}
\publication{Ji et al.}{\apj}{782}{95}{2014}
\end{publications}

%%%%%%%%%% 9. Condition of Closely-Related Past and Scheduled Observations %%%%%
%\relatobs{proposal ID}{title}{observational condition}{achievement(%)}
\begin{relationto}
\relatobs{None}{}{}{}
\relatobs{}{}{}{}
\relatobs{}{}{}{}
\relatobs{}{}{}{}
\relatobs{}{}{}{}
\end{relationto}

%%%%%%%%%% 10. Post-Observation Status and Publications %%%%%%%%%%
%\pastrun{year/month}{proposal ID}{PI name}{status:data reduction/analysis}{status:publication}
\begin{previoususe}
\pastrun{None}{}{}{}{}
\pastrun{}{}{}{}{}
\pastrun{}{}{}{}{}
\pastrun{}{}{}{}{}
\pastrun{}{}{}{}{}
\pastrun{}{}{}{}{}
\end{previoususe}

%%%%%%%%%% 11. Experience %%%%%%%%%%
\experience{PI Casey and CIs Ji, Ness, and Schlaufman are experts in high-resolution stellar spectroscopy, and CI Ho is an expert in low-resolution stellar spectroscopy. CIs Lattanzio, Karakas, Tout are experts on stellar evolution, nucleosynthesis, and binary stars. The remaining CIs are students. PI Casey, who will prepare and conduct the observations, has observed with the following telescope/instrument combinations:  Magellan/MIKE, VLT/UVES, Keck/HIRES, Gemini North/GMOS-N and Gemini South/GMOS-S (service mode), Mayall telescope/echelle, AAT/AAOmega, Automated Planet Finder (2.5 m) telescope (service), the ANU 2.3 m telescope, and about a month of commissioning time with the SkyMapper (imaging) telescope. Casey has written bespoke \texttt{Python} reduction packages for several of these telescope/instrument combinations, and many co-investigators have written software for the analysis of both low- and high-resolution spectra (Ho, Ness, and Casey; Ji, Ness, and Casey, respectively).}


%%%%%%%%%%%%
%% page 3 %%
%%%%%%%%%%%%

%%%%%%%%%% 12. Observing Run %%%%%%%%%%
%\run{instrument}{# of nights}{lunar phase}{preferred dates}{acceptable dates}{observing mode}
% "totnights" is calcurated automatically by ProMS. No need to fill \totnights{}.
\begin{observingrun}
\run{HDS}{2}{Grey}{29 Nov to 30 Nov}{1 Nov to 15 Dec}{Remote from Hilo}
\run{}{}{}{}{}{}
\run{}{}{}{}{}{}
\minnights{2}
\totnights{2}
\secondchoice{}
\orcomment{The PI, who will conduct the observations, has scheduling constraints from 14 Nov to 19 Nov. These dates are unavailable.}
\end{observingrun}

%%%%%%%%%% 13. Scheduling Requirements %%%%%%%%%%
\begin{schedule}
\scheduling{Request for remote observations at Hilo Base Facility. }
% Uncomment the following line if you want to make remote observations at Hilo. 
\remoteobs
% Uncomment the following line if you want to make remote observations at Mitaka. 
%\remotemtk
% Uncomment the following line if you want to make ToO observations. 
%\obstoo
% Uncomment the following line if you want to make time critial observations. 
%\obstcritical
\end{schedule}

%%%%%%%%%% 14. List of Targets %%%%%%%%%%
%\target{name}{RA}{DEC}{magnitude}
% Equinox is J2000.0 unless otherwise specified in Comments on Targets
% HH MM SS +/-DD MM SS or HH:MM:SS +/-DD:MM:SS are recommended for format of RA and Dec
\begin{targets}
% Candidates selected from mgk-candidates.csv
% and formatted into candidates.tex by candidates.py
% (candidates.tex required a little bit of text editing
\target{J00564998+3917229}{00:56:49.98}{+39:17:22.9}{V=12.6 $t_{\rm exp} = 2400$ s}
\target{J01030595+0434459}{01:03:05.95}{+04:34:45.9}{V=10.9 $t_{\rm exp} = 500$ s}
\target{J01194936+0634114}{01:19:49.36}{+06:34:11.4}{V=9.7 $t_{\rm exp} = 200$ s}
\target{J01303908+4048438}{01:30:39.08}{+40:48:43.8}{V=12.8 $t_{\rm exp} = 2400$ s}
\target{J03010142+5600423}{03:01:01.42}{+56:00:42.3}{V=12.2 $t_{\rm exp} = 2400$ s}
\target{J03245035+3524018}{03:24:50.35}{+35:24:01.8}{V=12.7 $t_{\rm exp} = 2400$ s}
\target{J03371425+4030151}{03:37:14.25}{+40:30:15.1}{V=12.8 $t_{\rm exp} = 2400$ s}
\target{J03445882+5929551}{03:44:58.82}{+59:29:55.1}{V=12.6 $t_{\rm exp} = 2400$ s}
\target{J03455201+5957393}{03:45:52.01}{+59:57:39.3}{V=13.6 $t_{\rm exp} = 3600$ s}
\target{J03482625+3501234}{03:48:26.25}{+35:01:23.4}{V=13.6 $t_{\rm exp} = 3600$ s}
\target{J03545585+5947303}{03:54:55.85}{+59:47:30.3}{V=12.3 $t_{\rm exp} = 2400$ s}
\target{J03565248+3259372}{03:56:52.48}{+32:59:37.2}{V=13.4 $t_{\rm exp} = 3600$ s}
\target{J03592829+5959328}{03:59:28.29}{+59:59:32.8}{V=13.0 $t_{\rm exp} = 3600$ s}
\target{J04034038+5639024}{04:03:40.38}{+56:39:02.4}{V=13.4 $t_{\rm exp} = 3600$ s}
\target{J04063497+4113549}{04:06:34.97}{+41:13:54.9}{V=12.8 $t_{\rm exp} = 2400$ s}
\target{J04081731+4632525}{04:08:17.31}{+46:32:52.5}{V=13.1 $t_{\rm exp} = 3600$ s}
\target{J04105545+5705426}{04:10:55.45}{+57:05:42.6}{V=13.4 $t_{\rm exp} = 3600$ s}
\target{J04190021+4222239}{04:19:00.21}{+42:22:23.9}{V=13.9 $t_{\rm exp} = 3600$ s}
\target{J04312148+4147572}{04:31:21.48}{+41:47:57.2}{V=13.8 $t_{\rm exp} = 3600$ s}
\target{J04341864+5352502}{04:34:18.64}{+53:52:50.2}{V=13.4 $t_{\rm exp} = 3600$ s}
\target{J04350382+5525146}{04:35:03.82}{+55:25:14.6}{V=13.2 $t_{\rm exp} = 3600$ s}
\target{J04440336+5432036}{04:44:03.36}{+54:32:03.6}{V=13.6 $t_{\rm exp} = 3600$ s}
\target{J04542486+5346285}{04:54:24.86}{+53:46:28.5}{V=13.9 $t_{\rm exp} = 3600$ s}
\target{J05055193+4937260}{05:05:51.93}{+49:37:26.0}{V=13.5 $t_{\rm exp} = 3600$ s}
\target{J05060674+4949250}{05:06:06.74}{+49:49:25.0}{V=12.9 $t_{\rm exp} = 2400$ s}
\target{J05074059+4929550}{05:07:40.59}{+49:29:55.0}{V=13.0 $t_{\rm exp} = 3600$ s}
\target{J05104605+5238483}{05:10:46.05}{+52:38:48.3}{V=12.9 $t_{\rm exp} = 2400$ s}
\target{J05251626+5859125}{05:25:16.26}{+58:59:12.5}{V=12.6 $t_{\rm exp} = 2400$ s}
\target{J05281458+5119124}{05:28:14.58}{+51:19:12.4}{V=13.8 $t_{\rm exp} = 3600$ s}
\target{J05292425+4943197}{05:29:24.25}{+49:43:19.7}{V=12.5 $t_{\rm exp} = 2400$ s}
\target{J05593508+1502521}{05:59:35.08}{+15:02:52.1}{V=12.8 $t_{\rm exp} = 2400$ s}
\target{J06014280+1323214}{06:01:42.80}{+13:23:21.4}{V=13.0 $t_{\rm exp} = 3600$ s}
\target{J06114417-0408451}{06:11:44.17}{-04:08:45.1}{V=13.0 $t_{\rm exp} = 3600$ s}
\target{J06142758+3315443}{06:14:27.58}{+33:15:44.3}{V=12.9 $t_{\rm exp} = 2400$ s}
\target{J06390857+3817280}{06:39:08.57}{+38:17:28.0}{V=13.6 $t_{\rm exp} = 3600$ s}
\target{J06413246+5804009}{06:41:32.46}{+58:04:00.9}{V=13.9 $t_{\rm exp} = 3600$ s}
\target{J07215618+2254053}{07:21:56.18}{+22:54:05.3}{V=12.5 $t_{\rm exp} = 2400$ s}
\target{J07291252+0734369}{07:29:12.52}{+07:34:36.9}{V=13.7 $t_{\rm exp} = 3600$ s}
\target{J07414665+1453305}{07:41:46.65}{+14:53:30.5}{V=11.8 $t_{\rm exp} = 1200$ s}
\target{J07484857+2106555}{07:48:48.57}{+21:06:55.5}{V=13.5 $t_{\rm exp} = 3600$ s}
\target{J07504312+2046580}{07:50:43.12}{+20:46:58.0}{V=11.4 $t_{\rm exp} = 1200$ s}
\target{J07512958+0241369}{07:51:29.58}{+02:41:36.9}{V=13.5 $t_{\rm exp} = 3600$ s}
\target{J09182549+1721145}{09:18:25.49}{+17:21:14.5}{V=13.8 $t_{\rm exp} = 3600$ s}
\target{J12003260+0244382}{12:00:32.60}{+02:44:38.2}{V=13.1 $t_{\rm exp} = 3600$ s}
\target{J16391001+1005453}{16:39:10.01}{+10:05:45.3}{V=13.9 $t_{\rm exp} = 3600$ s}
\target{J21391406+0255362}{21:39:14.06}{+02:55:36.2}{V=13.5 $t_{\rm exp} = 3600$ s}
\target{J21472125+0239587}{21:47:21.25}{+02:39:58.7}{V=11.9 $t_{\rm exp} = 1200$ s}
\target{J22011566-0014322}{22:01:15.66}{-00:14:32.2}{V=12.5 $t_{\rm exp} = 2400$ s}
\target{J22065778+2321383}{22:06:57.78}{+23:21:38.3}{V=13.1 $t_{\rm exp} = 3600$ s}
% no need to fill \targetflag
\targetflag{}
\targetcomment{}
\end{targets}


%%%%%%%%%%%%
%% page 4 %%
%%%%%%%%%%%%

%%%%%%%%%% 15. Observing Method and Technical Details %%%%%%%%%%
\begin{technicalinfo}
We propose to use the `Ra' setup with the default grating and a central wavelength of 5500\,\AA. We propose to use the atmospheric dispersion corrector (ADC In) with no image rotator, and the Messia5 acquisition system. With 2x1 binning (spatial x spectral), we expect readout time of 1 minute per exposure. The default echelle grating angle of 0.25\,$^\circ$ and detector rotation angle of $-1$\,$^\circ$ will be suitable for these observations. We propose to use the normal slit width of 0.4 arcseconds, and assume a seeing of 0.5 arcseconds.\\
 
These instrument settings will provide a spectral resolution of 90,000 from 5106\,\AA\ to 7787\,\AA. This wavelength coverage includes a number of spectral lines that will be essential for our spectroscopic analysis, allowing us to estimate detailed chemical abundances in order to understand the nature of stars with peculiar Mg/K abundance patterns. For example, the proposed configuration provides us wavelength coverage includes H-$\alpha$ line at 6562\,\AA, as well as atomic absorption lines of Li, $\alpha$-capture elements (Mg, Ca, Si, Ti), odd-Z elements (Na, Al, K), Fe-peak elements (V, Mn, Sc, Cu, Ni, Fe), and neutron-capture elements (Eu, Ba, Sr, Y, Zr, among others). The gap between the blue and red CCD, 6378\,\AA\ to 6509\,\AA\ in this configuration, does not include any stellar spectral lines that are essential for our analysis. Based on the technical details above, we have calculated exposure times for all candidates in order to achieve a minimum S/N ratio of 100 per pixel at the bluest wavelength. Experience has demonstrated that, given the high-resolution of 90,000, a S/N ratio of 100 per pixel will be sufficient to confirm the stellar parameters we derive from low-resolution spectra, and estimate detailed chemical abundances. If we find that isotopic ratios are required to differentiate hypotheses, the high spectral resolution and high  S/N ratios from these  observations will be  suitable to measure a limited number of isotopic ratios.\\

The exposure times listed in Section 13 for each candidate are calculated using the assumptions above, and requiring a S/N ratio of 100 per pixel at  the bluest wavelength. Exposure times will be varied at the telescope to take into account the precise V-band magnitude of each star, and the current seeing conditions. The candidates listed in Section 13 represent the brightest of our sample. We estimate average overheads of 5\,min per target. With these assumptions, we expect to be able to observe 29 to 30 of our brightest candidates in two December nights, assuming 10.25 hours of useful observing time per night. A sample size of $\sim$30 will be sufficient to differentiate between different hypotheses for the origin of Mg/K stars.\\

We have requested grey time as the moon is very close ($<40\,^\circ$) to most of our targets in bright time between the acceptable dates. If successful,  we specify a minimum allocation of two nights in order for our scientific objectives to be achieved. If we restricted ourselves to observing only the brightest candidates in order to make use of a single night allocation, the candidates we observe would not be representative in overall metallicity (the more metal-poor stars are fainter, requiring longer exposures), intrinsically biasing our inferences, and restricting our sample to at most 17 Mg/K stars.\\

%We will be using natural guide stars (NGS) that are bright ($R < 14$), and as such we conservatively estimate average overheads of 5\,min per target (including 3\,min for acquisition). With 

The radial velocities for all of our candidates are known from their LAMOST spectra. Under the assumption that all Mg/K candidates are single stars that are not experiencing any significant radial velocity excursions, we have preferentially selected candidates for high-resolution follow-up where the potassium lines at 7699\,\AA\ will be well-separated from the telluric A-band. However, we will observe a number of telluric standards throughout the observing run, at various airmasses, in order to allow for telluric removal and/or masking.
\end{technicalinfo}

%%%%%%%%%% 16. Instrument Requirements %%%%%%%%%%
\instruments{Standard SETUP-Ra for HDS with default grating and 0.4 arcsecond slit size.}

%%%%%%%%%% 17. Backup Proposal in Poor Conditions %%%%%%%%%%
\backup{Most of our targets are bright (all have $V < 14$) and can be observed in poor seeing. However, we also have a backup plan in the case of inclement weather. In separate work, we have applied the same search technique to identify lithium-rich giant stars from LAMOST data. If conditions are very poor, then we will observe bright ($V < 10$) lithium-rich giant stars selected from LAMOST, with a lithium-normal giant of identical stellar parameters for every lithium-rich giant, in order to perform the first ever differential analysis of lithium-rich giant stars. Our sample of lithium-rich giant stars are bright and span across all right ascensions, making them very suitable backup targets. The proposed setup (SETUP-Ra) for Mg/K stars is suitable for our lithium-rich giant analysis.}

%%%%%%%%%% 18. Public Data Archive of Subaru %%%%%
\begin{smoka}
% Uncomment the following line if you checked the public Subaru Telescope archive. 
\smokacheck
% If you propose observations in spite that there are available data in the public data archive, 
% please denote the reason in the following field. 
\smokacomment{We checked SMOKA for HDS data for every candidate and found no matches.}
\end{smoka} 

%%%%%%%%%% 19. Justify Duplications with the HSC SSP %%%%%%%%%%
\hscssp{N/A}


%%%%%%%%%%%%%
%% page 5+ %%
%%%%%%%%%%%%%

%%%%%%%%%% 14+. List of (more) Targets  %%%%%%%%%%
%*** (optional) ***
% If you have more targets, please uncomment the following 
% 3 lines and enter them here. 
%\begin{moretargets}
%\target{}{}{}{}
%\end{moretargets}

%%%%%%%%%% 17+. List of Backup Targets %%%%%%%%%%
%*** (optional) ***
% If you have backup targets decribed in entry 17, 
% please uncomment the following 3 lines and enter their data here, 
% using the same format as in entry 14. 
%\begin{backuptargets}
%\target{}{}{}{}
%\butargetcomment{}
%\end{backuptargets}


\end{document}

